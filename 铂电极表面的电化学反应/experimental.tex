\noindent \textbf{1、仪器与试剂} \par
仪器: CHI600B-03099d型电化学分析仪,上海辰华仪器公司;配套软件:CHI600B电化学工作站;三电极系统:参比电极为双盐桥饱和甘汞电极,辅助电极为铂片电极,雷磁公司,工作电极为铂圆盘电极,中国科学院南京土壤研究所\par 
试剂: 0.05 \mysi{mol.L^{-1}} \ce{H2SO4} 溶液; 0.1 \mysi{mol.L^{-1}} \ce{H2SO4} 溶液; 0.05 \mysi{mol.L^{-1}} 甲醇溶液. \par 
\noindent \textbf{2、实验内容} \par 
\noindent \textbf{(1)电极系统的清洁与活化}\par 
依次用自来水和去离子水充分清洗电解池和三电极系统,随后将三电极电解池与 CHI 电化学工作站相连:工作电极连接绿线;辅助电极连接红线;参比电极连接白线,向电解池中加入0.05 \mysi{mol.L^{-1}} \ce{H2SO4} 溶液至液面没过电极底部。\par 
以每秒3个气泡速度向电解池中通入\ce{N2},并开启磁力搅拌。通气搅拌3分钟至饱和后,在工作站中设置不同的电位区间开始循环扫描,寻找合适的电位窗口(氢区曲线不下沉、氧区曲线不上扬的前提下使区间尽可能宽)。经过多次尝试后得到最佳电位窗口为$-$0.28V\my~ 1.20V,后续活化及测量实验均在该窗口下进行。设置扫描速度为0.5 \mysi{V.s^{-1}},循环至观察到CV曲线的形状基本不变后停止活化。\par 
CV曲线稳定,且双电层区在0附近,说明电极活化完成。\par 
\noindent \textbf{(2)不同扫描速度下\ce{N2}饱和的\ce{H2SO4} 溶液的CV测试} \par 
活化完成后,停止磁力搅拌和通气体,分别设定扫描速度为0.5、0.2、0.1\mysi{V.s^{-1}},扫描已用\ce{N2}饱和的0.05 \mysi{mol.L^{-1}}硫酸溶液的CV曲线。 \par 
\noindent \textbf{(3)铂电极表面的氧还原反应(ORR)} \par 
以每秒3个气泡速度向电解池中通入\ce{O2},并开启磁力搅拌。通气搅拌5分钟至饱和后,分别在高速搅拌并以高流量通入\ce{O2}(每秒5个气泡)、停止搅拌并以低流量通入\ce{O2}(每秒1\my~ 2个气泡)、停止搅拌并关闭气体的条件下,扫描溶液的CV曲线,扫描速度均为0.1 \mysi{V.s^{-1}}。 \par 
\noindent \textbf{(4)铂电极表面的甲醇氧化反应(MOR)}\par 
等体积混合0.1 \mysi{mol.L^{-1}} 甲醇溶液和0.1 \mysi{mol.L^{-1}} \ce{H2SO4} 溶液,加入电解池中。开启搅拌并通入\ce{N2}至饱和后,以0.1 \mysi{V.s^{-1}}的扫描速度对溶液进行循环伏安扫描。\par 
\noindent \textbf{(5)电极系统的清洗}\par 
所有测量实验结束后,重新用自来水和去离子水清洗电极和电解池,在与活化实验相同的条件下再次测量\ce{N2}饱和的0.05 \mysi{mol.L^{-1}} \ce{H2SO4} 溶液的CV曲线。曲线与实验开始时活化后的CV曲线几乎重合,说明电极系统和电解池已经清洗干净。复原实验仪器。 \par