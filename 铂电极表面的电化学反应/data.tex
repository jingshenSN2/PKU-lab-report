\noindent \textbf{1、扫描速度对CV曲线的影响与电极反应的可逆性} \par
根据实验部分2(2),绘制了不同扫描速度下的\ce{N2}饱和的0.05 \mysi{mol.L^{-1}} \ce{H2SO4} 溶液的CV曲线,如图1所示(测量10个Segment并取Segment\ 9和Segment\ 10作图)。\par
\myfig{N2_tot_G.eps}{图1\ 不同扫描速度下\ce{N2}饱和的0.05 \mysi{mol.L^{-1}} \ce{H2SO4} 溶液的CV曲线}
从图像中可以看出,随着扫描速度的增加,双电层均在0附近,特征峰位置基本不变,峰电流增大,与循环伏安法的原理相符。在CV曲线中能较明显地找到五个氧化还原峰,分别为电位在0.4 V附近的含氧物种的还原峰,和位于$-$0.2\my~ $-$0.05 V的电位区间的四个H的吸附/脱附峰。扫描速度越大,由于峰电流增大,出峰越明显。根据这些特征峰的峰电流、峰电位值可以研究电极反应的反应性。相关数据列于表1中,其中$i_{p}$为氧区还原峰电流,$E_{p1}$、$E_{p2}$、$E_{p1}'$、$E_{p2}'$分别为氢的强吸附峰、弱吸附峰、弱脱附峰、强脱附峰的峰电位。 \par
\begin{table}[h]
	\centering
	\begin{center}
	\tupian{表1\ 不同扫描速度下\ce{N2}饱和的\ce{H2SO4} 溶液CV曲线的部分参数}
	\end{center}
	\label{tab:width:fixed}
	\sisetup{ 
		 table-number-alignment = center,
		 table-column-width = 2cm , 
	}
	\begin{tabular}
	{
		S
		S[fixed-exponent = -5]
		S
		S
		S
		S
	}
 	\toprule
 	{扫速(\mysi{V.s^{-1}})} & {$i_{p}$ (\num{e-5}A)} & {$E_{p1}$(V)} & {$E_{p2}$(V)} & {$E_{p1}'$(V)} & {$E_{p2}'$(V)} \\ 
 	\midrule 
	0.5&	5.138&	-0.068&	-0.235&	-0.212&	-0.047\\
	0.2&	2.406&	-0.064&	-0.229&	-0.215&	-0.049\\
	0.1&	1.359&	-0.063&	-0.227&	-0.216&	-0.050\\
	\bottomrule
	\end{tabular}
\end{table}
\zhengwen
由表中数据可见,随着扫描速度变慢,对应吸附/脱附峰的差值$\Delta E_{p}$从0.02V减小至0.01V,且峰对称性变高,说明反应基本可逆。另外用氧区还原峰电流对扫描速度做双对数图,具有很好的线性关系,斜率为0.82,接近于1,说明此条件下的含氧物种还原峰具有可逆吸附波的特征。\par

\noindent \textbf{2、铂电极电化学活性面积的测定} \par
在0.1 \mysi{V.s^{-1}}的扫描速度下,停止搅拌且关闭气体,绘制被\ce{N2}饱和的硫酸溶液的CV曲线,如图3所示。根据氢脱附区的曲线下围面积(以右侧双电层的最低点为基线),用Origin给出其积分面积为\num{8.63e-7} \mysi{V.A}。根据电量$Q$的定义,有
$$Q=\int{I\mathrm{d} t}=\int{\frac{I}{v}\mathrm{d} E}=\frac{1}{v}\int{I\mathrm{d}E}=\frac{Area}{v}$$\par 
因此$Q=\num{8.63e-6}$C,多晶Pt表面满单层H原子脱附量为0.21 \mysi{mC.cm^{-2}},故Pt电极的电化学活性面积为0.041 \mysi{cm^{2}}。\par 
\myfig{DesorpArea_G.eps}{图2\ 被\ce{N2}饱和的硫酸溶液的CV曲线中氢脱附区的积分面积图}

\noindent \textbf{3、氧还原反应(ORR)的CV曲线} \par
在0.1 \mysi{V.s^{-1}}的扫描速度下,停止搅拌且关闭气体,分别绘制被\ce{N2}和\ce{O2}饱和的硫酸溶液的CV曲线,并求出差值,结果如图3所示(测量8个Segment并取Segment\ 7和Segment\ 8作图)。\par
由图可见,当电位大于0.6V时,介质中的\ce{O2}几乎不发生额外的化学行为,CV曲线与\ce{N2}的基本重合。当电位回扫到0.4V时,还原峰电流迅速增大,此时电极上发生了ORR反应,以差值电流超过\num{2.0e-7}A为界限,ORR发生的起始电位为0.532V。\par 
控制扫描速度为0.1\mysi{V.s^{-1}},分别在高速搅拌并以高流量通入\ce{O2}(每秒5个气泡)、停止搅拌并以低流量通入\ce{O2}(每秒1\my~ 2个气泡)、停止搅拌并关闭气体的条件下,测定溶液的CV曲线,如图4所示。\par 
同样在电位大于0.6V时,三条曲线几乎重合,说明此时溶液的电化学响应基本不变。但在发生ORR的部分,由于溶液中对流、传质情况的不同而产生的较大差别。对流越剧烈,电极附近的\ce{O2}和\ce{H^{+}}浓度涨落就越大,使得电流变得不稳定。\par 
\myfig{N2O2diff_G.eps}{图3\ 被\ce{N2}和\ce{O2}饱和的0.05 \mysi{mol.L^{-1}} \ce{H2SO4} 溶液的CV曲线及差值曲线}
\myfig{O2_tot_G.eps}{图4\ 不同搅拌速度下被\ce{O2}饱和的0.05 \mysi{mol.L^{-1}} \ce{H2SO4} 溶液的CV曲线}

\noindent \textbf{4、甲醇氧化反应(MOR)的CV曲线} \par
在0.1 \mysi{V.s^{-1}}的扫描速度下,停止搅拌且关闭气体,分别绘制被\ce{N2}饱和的硫酸溶液和甲醇/硫酸混合溶液的CV曲线,并求出差值,结果如图5所示(测量8个Segment并取Segment\ 7和Segment\ 8作图)。\par
被\ce{N2}饱和的甲醇/硫酸溶液的CV曲线较为复杂,与原始的\ce{N2}饱和的硫酸溶液的CV曲线相差较大,正扫时存在两个甲醇氧化峰,分别位于0.5V和1.2V右侧,回扫时除ORR还原峰外,还存在一个0.3V左右的氧化峰,意味着在回扫阶段依然是甲醇的氧化反应而不是还原反应。取阳极区段中差值电流为0的点,可以读出MOR发生的起始电位为0.084V。
\myfig{MeOHN2diff_G.eps}{图5\ 被\ce{N2}饱和的硫酸溶液和甲醇/硫酸混合溶液的CV曲线及差值曲线}

\noindent \textbf{5、铂阴阳极的直接甲醇燃料电池的输出功率-输出电压曲线} \par
将铂电极上发生的ORR和MOR组合成原电池。在等电流的情况下,输出电压为ORR电位减去MOR电位,输出功率为输出电压乘以电流,即
$$U_{cell}=E_{ORR}-E_{MOR},P_{cell}=i\cdot U_{cell}$$\par 
分别将扣除\ce{N2}背景后的氧气和甲醇的CV曲线中ORR和MOR的区段绘制在图中,结果如图6所示。由于电池两极中电流的方向相反,因此需要将ORR曲线向上翻转,即电流取绝对值。为保证电池能够正常输出电流,氧气应在正极还原,甲醇应在负极氧化,输出电压为正值,因此只有图中的阴影区域才是这个电池的有效工作区域。\par
\myfig{EffArea_G.eps}{图6\ 铂阴阳极的直接甲醇燃料电池的有效工作区域}
首先提取出阴影区域的电位电流信息,交换$xy$轴后,并将曲线分成上下两部分。用Origin的插值功能将原本不均匀分布的$x$值填充为等间隔的值,以方便后续求差值。用两组曲线上插值后的点两两相减,求出对应的输出电压$U_{cell}$和输出功率$P_{cell}$。作出电池的输出功率-输出电压曲线,如图7所示。\par
\myfig{Power_G.eps}{图6\ 铂阴阳极的直接甲醇燃料电池的有效工作区域}
从图中可以看出,铂阴阳极的直接甲醇燃料电池的输出功率随着输出电压的升高,先升高后降低,在输出电压约为0.145V时达到最大值\num{8.17e-7} W。\par

