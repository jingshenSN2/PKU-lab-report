\noindent \textbf{1、误差分析和实验的改进} \par
在本次实验中,所有的数据和结论均是由循环伏安曲线图直接或间接的得出,因此误差的来源即为循环伏安测试的过程。在使用电化学工作站进行循环伏安测试时,电流的灵敏度参数的选取很重要,当电解质溶液的导电性较好时,为防止电流过大超过量程,需要选取较大的灵敏度参数,即更低的灵敏度;而如果导电性较差电流很小时,就需要选取较小的灵敏度参数,提高仪器检测电流的灵敏度,否则可能出现实验数据波动较大,精度较差等问题。\par 
\noindent \textbf{2、本次实验中的重大失误} \par
在本次实验中,由于预习不够充分,没能准确理解实验讲义中“在实验过程中参比电极的二次盐桥中加 入的始终是 0.05 M 硫酸水溶液;实验结束后更换为去离子水”的含义。在实验开始时没有将二次盐桥中的去离子水替换为硫酸,仅在测试结束后换成了去离子水。\par 
从参比电极的结构角度考虑,如果二次盐桥内放去离子水,且长时间放置,会使得内盐桥中的一部分氯化钾扩散至二次盐桥中,这会在一定程度上影响参比电极的电导,可能使得其上有电流流过,造成工作电极和辅助电极间电位、电流的测量结果不准确。因而实验中的CV图均不够准确,例如双电层略微向下偏离0水平,氢区存在双线电流的情况。\par 
今后使用有二次盐桥的电极时,应注意及时将盐桥内的电解液换成与电解池中溶质种类、浓度均相同的溶液。\par
\noindent \textbf{3、实验结论} \par
本实验用循环伏安法研究了不同扫描速度下\ce{N2}饱和的0.05 \mysi{mol.L^{-1}} \ce{H2SO4} 溶液中铂电极的电化学响应性质,证明了含氧物种还原峰是可逆吸附波,铂电极的电化学活性面积为0.041 \mysi{cm^{-2}}。测定了\ce{N2}饱和的甲醇/硫酸混合溶液以及不同搅拌速度下\ce{O2}饱和的0.05 \mysi{mol.L^{-1}} \ce{H2SO4} 溶液的CV曲线,得出ORR发生的起始电位为0.532V,MOR发生的起始电位为0.084V,证明在对流剧烈的情况下,ORR反应的电流不确定性显著增大。结合MOR和ORR的CV曲线,绘制出铂阴阳极的直接甲醇燃料电池的输出功率-输出电压曲线,得到输出电压约为0.145V时,电池输出功率达到最大值\num{8.17e-7} W。\par 